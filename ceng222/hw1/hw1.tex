\documentclass[12pt]{article}
\usepackage[utf8]{inputenc}
\usepackage{float}
\usepackage{amsmath}


\usepackage[hmargin=3cm,vmargin=6.0cm]{geometry}
\topmargin=-2cm
\addtolength{\textheight}{6.5cm}
\addtolength{\textwidth}{2.0cm}
\setlength{\oddsidemargin}{0.0cm}
\setlength{\evensidemargin}{0.0cm}
\usepackage{indentfirst}
\usepackage{amsfonts}

\begin{document}

\section*{Student Information}

Name : Emre Geçit\\

ID : 2521581\\


\section*{Answer 1}
\subsection*{a)} The probability that at least one of the balls is white is simply 1 - P(no balls are white).
P(no balls are white) is equivalent to P(all balls are black). The events of choosing a ball from each box are independent events, so the probability of choosing all the balls black is: \\

P(no balls are white) = $\frac{8}{10} \cdot \frac{11}{15} \cdot \frac{9}{12}$ = 0.44 

P(at least one of the balls is white) = 1 - P(no balls are white) = 0.56
\subsection*{b)} P(all balls are white) = $\frac{2}{10} \cdot \frac{4}{15} \cdot \frac{3}{12}$ = 0.01333
\subsection*{c)}
P(choosing two white balls from the first box) = $\frac{2\cdot 1}{10\cdot 9}$ = 0.0222

P(choosing two white balls from the second box) = $\frac{4\cdot 3}{15\cdot 14}$ = 0.0571

P(choosing two white balls from the third box) = $\frac{3\cdot 2}{12\cdot 11}$ = 0.0455\\\\
As can be seen, the probability of choosing two white balls randomly from second box is larger than the others. Choosing the third box is the best decision.
\subsection*{d)}At each step, choosing the box with the highest rate of white balls is the optimal strategy. At first, the rates are 0.2, 0.27, 0.25. First, we choose from the second box. After choosing from the second box, the rate of white balls drops to 3/14 = 0.21. It will no longer be favorable to choose from the second box, so we continue with the box with the highest white ball rate, the third box. So, the answer is second box and then the third box. In fact, the order isn't important.\newpage
\subsection*{e)}Let $n_i$ denote the number of white balls choosen from i'th box and let T denote the number of total balls choosen. Then E(T) = E($n_1+n_2+n_3$)\\

= E($n_1$) + E($n_2$) + E($n_3$)

= 0.2 + 0.267 + 0.25

= 0.717
\subsection*{f)}Let $B_i$ denote the event where the ball is choosen from $i_th$ box and W denote the event choosen ball is white.\\\\
$P(B_1|W) = \frac{P(W|B_1)\cdot P(B_1)}{P(W)}\\
P(B_1)=1/3\\
P(W) = P(W|B_1)\cdot P(B_1) + P(W|B_2)\cdot P(B_2) + P(W|B_3)\cdot P(B_3) = 1/5 * 1/3 + 4/15 * 1/3 + 1/4 * 1/3 = 0.239\\
P(B_1|W) = \frac{P(W|B_1)\cdot P(B_1)}{P(W)} = 0.067 / 0.239 = 0.279
$
\section*{Answer 2}
\subsection*{a)}Let's denote some events with letters:\\
R: Ring is destroyed\\
S: Sam is corrupted\\
F: Frodo is corrupted\\\\
a)\\P(R|S') = 0.9\\P(R|S) = 0.5\\P(S) = 0.1\\P(S|R) = ?\\\\
$P(S|R) = \frac{P(R|S)\cdot P(S)}{P(R)}\\
P(R) = P(R|S)\cdot P(S) + P(R|S')\cdot P(S') = 0.5\cdot 0.1 + 0.9\cdot 0.9 = 0.86\\
P(S|R) = \frac{0.5\cdot 0.1}{0.86} = 0.0581$\\\\
\subsection*{b)}b)\\P(F) = 0.25\\P(R|F, S') = 0.2\\P(R|F', S') = 0.9\\P(R|F, S) = 0.05\\P(F, S| R) = ?\\
$
P(F, S| R) = \frac{P(R| F, S)\cdot P(F, S)}{P(R)}\\
= \frac{0.05 * 0.25 * 0.1}{0.86} = 0.00145$

\section*{Answer 3}
\subsection*{a)} Probability that there are 4 snowy days in total = P(2, 2) + P(3, 1) = 0.32
\subsection*{b)} If these two variables are independent then for all a and i in the domain, P(A = a, I = i) = P(A = a) * P(I = i). If for any case, this equality does not hold, then these variables will be dependent. Let's check if this equality holds for all cases.\\\\
P(1, 1) = 0.18, P(A = 1) * P(I = 1) = 0.3 * 0.6 = 0.18\\
P(1, 2) = 0.12, P(A = 1) * P(I = 2) = 0.3 * 0.4 = 0.12\\
P(2, 1) = 0.3, P(A = 2) * P(I = 1) = 0.5 * 0.6 = 0.3\\
P(2, 2) = 0.2, P(A = 2) * P(I = 2) = 0.5 * 0.4 = 0.2\\
P(3, 1) = 0.12, P(A = 3) * P(I = 1) = 0.2 * 0.6 = 0.12\\
P(3, 2) = 0.08, P(A = 3) * P(I = 2) = 0.2 * 0.4 = 0.08\\\\
Since for all cases the condition for independency holds, these two variables are independent.

\end{document}