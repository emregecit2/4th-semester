\documentclass[12pt]{article}
\usepackage[utf8]{inputenc}
\usepackage{float}
\usepackage{amsmath}


\usepackage[hmargin=3cm,vmargin=6.0cm]{geometry}
\topmargin=-2cm
\addtolength{\textheight}{6.5cm}
\addtolength{\textwidth}{2.0cm}
\setlength{\oddsidemargin}{0.0cm}
\setlength{\evensidemargin}{0.0cm}
\usepackage{indentfirst}
\usepackage{amsfonts}

\begin{document}

\section*{Student Information}

Name : Emre Geçit\\

ID : 2521581\\



\section*{Answer 1}
\paragraph{a)}
The mean of the data can be easily calculated to be 16.96

$\bar{X} = 16.96$

$1-\alpha = 0.90 \implies \alpha = 0.1$

$z_{\alpha/2} = z_{0.05} = 1.645$

Confidence interval of $\%90$ is $\bar{X} \pm z_{0.05}\dfrac{3}{\sqrt{10}} \Leftrightarrow [15.4, 18.5]$

$1-\alpha = 0.99 \implies \alpha = 0.01$

$z_{\alpha/2} = z_{0.005} = 2.576$

Confidence interval of $\%99$ is $\bar{X} \pm z_{0.005}\dfrac{3}{\sqrt{10}} \Leftrightarrow [14.5, 19.4]$
\paragraph{b)}
$1-\alpha = 0.98 \implies \alpha = 0.02$

$z_{\alpha/2} = z_{0.01} = 2.326$
\begin{center}
$n \geq (\frac{z_{0.01}\cdot\sigma}{\delta})^2$
\end{center}
\begin{center}
    $n \geq (\frac{2.326\cdot3}{1.55})^2$
\end{center}
\begin{center}
    $n \geq 20.267...$
\end{center}

The least n satisfying this inequality is $n = 21$

\newpage
\section*{Answer 2}
\paragraph{a)}
These are not enough in order to make meaningful comments about the distribution of the data. We also need the standard deviation of the data.
\paragraph{b)}
We test the null hypothesis $H_0: \mu = 7.5$ against a one-sided left-tail alternative $H_A: \mu < 7.5$, because we are only interested to know if the mean of rating $\mu$ is less than 7.5.

Step 1: Test statistic. We are given $\sigma = 0.8$, $n = 256$, $\alpha = 0.05$, $\mu_0 = 7.5$, and from the sample $\bar{X}=7.4$. The test statistic is
\[
    Z = \frac{\bar{X} - \mu_0}{\sigma/\sqrt{n}} = \frac{7.4 - 7.5}{0.8/\sqrt{256}} = -2.
\]

Step 2: Acceptance and rejection regions. The critical value is
\[
    z_\alpha = z_{0.05} = 1.645.
\]

With the left-tail alternative, we
\begin{center}
    reject $H_0$ if $Z < -1.645$.

    accept $H_0$ if $Z \geq -1.645$.
\end{center}

Our test statistic $Z = -2$ belongs to the rejection region; therefore, we reject the null hypothesis.

Restaurant A would not be in my list of candidate restaurants to order food from.
\paragraph{c)}
$\sigma = 1.0$, $n = 256$, $\alpha = 0.05$, $\mu_0 = 7.5$, $\bar{X}=7.4$
\[
    Z = \frac{\bar{X} - \mu_0}{\sigma/\sqrt{n}} = \frac{7.4 - 7.5}{1.0/\sqrt{256}} = -1.6
\]

Our test statistic $Z = -1.6$ belongs to the acceptance region; therefore, we accept the null hypothesis.

In this case, restaurant A would be in my list of candidate restaurants.
\paragraph{d)}
I will consider placing an order from a restaurant if and only if the rating of that restaurant is not significantly lower than 7.5. There is no need for resorting to a statistical test for values restaurants with a rating greater than or equal to 7.5, a greater rating will never be rejected. We should resort statistical tests when the rating is lower than 7.5.
\newpage
\section*{Answer 3}
\paragraph{a)}
We test the null hypothesis $H_0: \mu_A = \mu_B + 90$ against one sided left-tail alternative\\$H_A: \mu_A < \mu_B + 90$.

Since the variances are equal, we can use the pooled sample variance formula.
\[
    s_p^2 = \dfrac{(n-1)s_X^2 + (m-1)s_Y^2}{n+m-2} = \dfrac{19\cdot27.04 + 31\cdot519.84}{50} = 332.576
\]

\[
    s_p \simeq 18.237
\]

Test statistic t
\[
    t = \frac{\bar{X} - \bar{Y} - D}{s_p\sqrt{\dfrac{1}{n}+\dfrac{1}{m}}} = \frac{211 - 133 - 90}{18.237\sqrt{\dfrac{1}{20}+\dfrac{1}{32}}} \simeq -2.31
\]

$t_\alpha = t_{0.01} = 2.403$ ($n + m - 2 = 50$ degrees of freedom)

Since the alternative hypothesis is left-tailed
\begin{center}
    reject $H_0$ if $t < -2.403$.

    accept $H_0$ if $t \geq -2.403$.
\end{center}

Our test statistic $t = -2.31$ is greater than the critical value $-t_\alpha = -2.403$; therefore, we accept the null hypothesis.

The researcher can claim that the computer B provides a 90-minute or better improvement.
\paragraph{b)} Use the same hypotheses from part a.

Test statistic t
\[
    t = \frac{\bar{X} - \bar{Y} - D}{\sqrt{\dfrac{s_X^2}{n}+\dfrac{s_Y^2}{m}}} = \frac{211 - 133 - 90}{\sqrt{\dfrac{27.04}{20}+\dfrac{519.84}{32}}} \simeq -2.86
\]

Degrees of freedom v (Satterthwaite's approximation)
\[
    v = \dfrac{(\dfrac{s^2_X}{n}+\dfrac{s^2_Y}{m})^2}{\dfrac{s^4_X}{n^2(n-1)}+\dfrac{s^4_Y}{m^2(m-1)}} = \dfrac{(\dfrac{27.04}{20}+\dfrac{519.84}{32})^2}{\dfrac{731.1616}{400\cdot19}+\dfrac{270233.6256}{1024\cdot31}} \simeq 36
\]
$t_{0.01} = 2.434$ (36 degrees of freedom)

Since the hypothesis is left-tailed
\begin{center}
    reject $H_0$ if $t < -2.434$.

    accept $H_0$ if $t \geq -2.434$.
\end{center}

Our test statistic $t = -2.86$ is less than the critical value $-t_{0.01} = -2.434$; therefore, we reject the null hypothesis.

In this case, the researcher cannot claim that the computer B provides a 90-minute or better improvement.
\end{document}

